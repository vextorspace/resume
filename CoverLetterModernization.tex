\documentclass[letter,12pt]{article}
\usepackage[utf8]{inputenc}
\usepackage[T1]{fontenc}
\usepackage{geometry}
\usepackage{amssymb}
\usepackage{enumitem}
\usepackage{hyperref}
\usepackage{ebgaramond}
\usepackage{textcomp}
\geometry{left=.75in, right=.75in, top=.75in, bottom=.75in}

\begin{document}

%======================
%   Header
%======================
\begin{center}
    {\Huge \textbf{Douglas Ronne}}\\
    \vspace{2mm}
    1068 Fir St SE| Olympia, WA, 98501\\
    \href{mailto:vextorspace@gmail.com}{vextorspace@gmail.com} | (360) 797-2009\\
    \href{www.linkedin.com/in/douglas-ronne-7133272a}{linkedin.com/in/douglas-ronne-7133272a} | \href{https://github.com/vextorspace}{github.com/vextorspace}
\end{center}

Dear Hiring Manager,
I am excited to apply for a position with Cognizant. I love working with legacy software. I began this when I started working at Protocase and took over their legacy CAD software. My first attempt was to simply rewrite it after I found it hard to work with. I thought I could rewrite it in a way that was easier to deal with. This was true for a while, then my nice fresh software became legacy software over time. I started reading books on legacy software and refactoring and test driven development. I began practicing these techniques and was able to run a team that continued to modify and grow this software without rewriting it. This was tremendously satisfying. I have learned too that sometimes you can rewrite something sucessfully if you have the time and the support.

I also have worked with a lot of lower level languages like C, assembly, Ada, Pascal. I believe this would help me when translating old software to new. I am not afraid of the more technical languages.

Finally, I did a lot of customer outreach where I would try to learn what their real problems were and how I could solve them. I have always been very empathetic to others' issues and find it easy to relate.

Thank you for considering me for your role.
\\
Sincerely,
--Douglas Ronne
\end{document}
